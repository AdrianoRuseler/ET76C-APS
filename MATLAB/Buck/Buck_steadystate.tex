% Tabela com o ponto de opera��o do conversor
\begin{table}[!ht]
\centering
\caption{Ponto de opera��o do conversor Buck referente ao registro acad�mico de n�mero $1234567$}
\label{tab:steadystate}
\begin{tabular}{@{}ccc@{}}
\toprule
\textbf{S�mbolo} & \textbf{Descri��o} & \textbf{Valor}\\ \midrule
$G$ & Ganho est�tico & \SI{0.22222}{}\\
$D$ & Raz�o c�clida  & \SI{22.2222}{\%}\\
$I_0$ & Corrente m�dia na carga  & \SI{35}{\A} \\
$I_{L_0}$ & Corrente m�dia no indutor & \SI{35}{\A} \\
$V_C$ & Tens�o de controle  & \SI{0.22222}{\V} \\
$V_{CM}$ & Tens�o m�xima de controle  & \SI{1}{\V} \\
$V_{Cm}$ & Tens�o m�nima de controle  & \SI{0}{\V} \\
$R_a$ & Resist�ncia de medi��o & \SI{270}{\kilo\ohm} \\
$R_b$ & Resist�ncia de medi��o & \SI{1.2}{\kilo\ohm} \\
$H_v$ & Ganho de medi��o (tens�o) & \SI{4.4248}{\milli\V\per\V} \\
$R_s$ & Resist�ncia shunt & \SI{0.1}{\ohm} \\
$H_i$ & Ganho de medi��o (corrente) & \SI{0.1}{\A\per\A} \\
\bottomrule
\end{tabular}
\end{table}

